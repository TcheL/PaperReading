\renewcommand{\pmk}{ZhangW\_2006\_GJI\_Traction image method}
\renewcommand{\prf}{Modelling/\pmk.pdf}
\renewcommand{\pti}{Traction image method for irregular free surface boundaries
in finite difference seismic wave simulation}
\renewcommand{\pay}{Wei Zhang, Xiaofei Chen, 2006}
\renewcommand{\pjo}{Geophys. J. Int.}
\renewcommand{\pda}{2016/10/29 Sat.}

\section{\pinfo}
\subsection{Introduction}
\begin{enumerate}[\hspace{10mm}*]
  \item Use finite difference method (FDM) in rupture dynamics of
    earthquake source: Madariaga, 1976; Andrews, 1976a \& 1976b;
    Olsen \etal, 1997; Madariaga \etal, 1998; Cruz-Atienza and Virieux, 2004.
  \item Use FDM in seismic wave propagation in complex heterogeneous media:
    Boore, 1972; Kelly \etal, 1976; Bayliss \etal, 1986; Virieux, 1984 \& 1986;
    Levander, 1988; Graves, 1996; Dai \etal, 1995; Zahradnik, 1995.
  \item Free surface conditions: Jih \etal, 1988; Oprsal and Zahradnik, 1999;
    Ohminato and Chouet, 1997; Robertsson, 1996; Hestholm and Ruud, 1994 \& 1998.
  \item \sline
  \item Free surface conditions for a planar surface:
    Gottschammer and Olsen, 2001; Kristek \etal, 2002.
  \item Vacuum method: Boore, 972; Graves, 1996.
  \item Characteristic variables method: Bayliss \etal, 1986.
  \item Adjusted FD approximations (AFDA) technique: Kristek \etal, 2002.
  \item Stress image method: Levander, 1988; Graves, 1996.
  \item \sline
  \item Extend the stress image method with staircase approximation to
    the general topographic problem in the second-order accurate
    staggered finite difference scheme: Ohminato and Chouet, 1997.
  \item Implement the stress image method with staircase approximation to
    the irregular surface in the fourth-order staggered scheme:
    Robertsson, 1996; Pitarka and Irikura, 1996.
  \item \sline
  \item Vertical grid mapping to match the computational grids with
    the surface topography in staggered finite difference schemes:
    Hestholm and Ruud, 1994 \& 1998.
  \item \sline
  \item Boundary-conforming grid in seismic wave simulation with
    pseudospectral method: Fornberg, 1988.
  \item Numerical grid generation: Thompson \etal, 1985.
  \item The original MacCormack scheme with 2nd-order accurate
    in both time and space: MacCormack, 1969.
  \item Extend MacCormack scheme to 2nd-order accurate in time and
    4th-order accurate in space (2-4 MacCormack scheme):
    Gottlieb and Turkel, 1976.
  \item Introduce 2-4 MacCormack scheme into seismic wave modelling:
    Bayliss \etal, 1986 (implement with an operator splitting).
  \item Use 2-4 MacCormack splitting scheme in seismic wave problems:
    Xie and Yao, 1988; Tsingas \etal, 1990; Vafidis \etal, 1992;
    Dai \etal, 1995.
  \item High-accuracy MacCormack schemes with the DRP/opt MacCormack scheme:
    Hixon, 1997.
  \item DRP (dispersion relation preserving) methodology: Tam and Webb, 1993.
  \item 4-6 LDDRK (low dispersion and dissipation Runge-Kutta) scheme:
    Hu \etal, 1996.
  \item 4/4 compact MacCormack scheme: Hixon and Turkel, 2000.
  \item Treat the discontinuous interior interfaces by effective parameters
    (arithmetic average or harmonic average): Moczo \etal, 2002.
  \item The approximated delta function by Herrmann pseudo-delta functions:
    Herrmann, 1979; Wang \etal, 2001.
  \item The split-field perfectly matched layer (PML) approach:
    \Berenger, 1994; Marcinkovich and Olsen, 2003.
\end{enumerate}

\subsection{DRP/opt MacCormack scheme}
In the DRP scheme, the forward and backward partial difference operators are:
\[ \hat W_i^F=\frac{1}{\Delta x}\sum_{j=-1}^3a_jW_{i+j} \]
\[ \hat W_i^B=\frac{1}{\Delta x}\sum_{j=-1}^3-a_jW_{i-j} \]
where the expansion coefficients are:
$a_{-1}=-0.30874,a_0=-0.6326,a_1=1.2330,a_2=-0.3334,a_3=0.04168$
and these coefficients are obtained by minimizing the dissipation error
at eight points or more per wavelength.

\subsection{Compact MacCormack scheme}
The 4/4 compact MacCormack scheme is:
\[ \hat W_{j-1}^B+2\hat W_j^B=\frac{1}{2\Delta x}(W_{j+1}+4W_j-5W_{j-1}) \]
\[ 2\hat W_j^F+\hat W_{j+1}^F=\frac{1}{2\Delta x}(5W_{j+1}-4W_j-W_{j-1}) \]
where $\hat W_j^F$ and $\hat W_j^B$ denote the forward and
backward difference operators.

\subsection{Interior interface conditions}
Treat the discontinuous interior interfaces by effective parameters,
the density by arithmetic average:
\[ \rho_{ij}=\frac{1}{\Delta S}\int_{i-\nicefrac{1}{2}}^{i+\nicefrac{1}{2}}\int_{j-\nicefrac{1}{2}}^{j+\nicefrac{1}{2}}\rho dxdy \]
and the \Lame parameters by harmonic average:
\[ \frac{1}{\mu_{ij}}=\frac{1}{\Delta S}\int_{i-\nicefrac{1}{2}}^{i+\nicefrac{1}{2}}\int_{j-\nicefrac{1}{2}}^{j+\nicefrac{1}{2}}\frac{1}{\mu}dxdy \]
\[ \frac{1}{\lambda_{ij}}=\frac{1}{\Delta S}\int_{i-\nicefrac{1}{2}}^{i+\nicefrac{1}{2}}\int_{j-\nicefrac{1}{2}}^{j+\nicefrac{1}{2}}\frac{1}{\lambda}dxdy \]

% vim:sw=2:wrap
