\renewcommand{\pmk}{Plessix\_2010\_SEG\_Application to land data set}
\renewcommand{\prf}{FWI/\pmk.pdf}
\renewcommand{\pti}{Application of acoustic full waveform inversion
to a low-frequency large-offset land data set}
\renewcommand{\pay}{Ren\'{e}-Edouard Plessix, Guido Baeten and Jan Willem de Maag \etal, 2010}
\renewcommand{\pjo}{SEG 2010 Annual Meeting}
\renewcommand{\pda}{2016/10/3 Mon.}

\section{\pinfo}
\subsection{Introduction}
\begin{enumerate}[\hspace{10mm}*]
  \item Proposing of full waveform inversion: Tarantola, 1987.
  \item 3D real marine examples: Plessix, 2009; Sirgue \etal, 2009; Vigh \etal, 2009.
  \item \sline
  \item Low frequencies and large offsets mitigate the sensitivity to the initial mdoel:
    Bunks \etal, 1995; Pratt, 1999.
  \item FWI can update the long spatial wavelengths of velocity: Gauthier \etal, 1986; Pratt, 1999.
  \item Apply to land data sets: Ravaut \etal, 2004; Brenders and Pratt, 2004.
    (Attenuate the elastic effects by focusing on the first breaks with windowing technique)
  \item \sline
  \item Solve the wave equation in the frequency domain: Plessix, 1997.
  \item The width of the valleys of the least-squares misfit
    is inversely proportional to frequency: Bunks \etal, 1995.
  \item Overlap the frequencies between scales to better retain the velocity updates
    of the low frequency scales: Brossier \etal, 2009.
\end{enumerate}

\subsection{Full waveform inversion}
The misfit function
\[ J_f(m)=\frac{1}{2}||W(c-d)||^2 \]
with a frequency $f$, the velocity field $m$, the modeled data $c$, the observed data $d$
and a data weighting matrix $W$ which is a diagonal matrix where the diagonal elements are $h^\beta$
with the offset $h$ and a coefficient $\beta$ generally between $0$ and $2$.

Minimize with the quasi-Newton algorithm
\myidx{Inversion}{Iteration}{quasi-Newton algorithm}
\[ m_{k+1}=m_k-\alpha_kB_k\nabla_mJ_f(m_k) \]
with the step length $\alpha_k$ and the approximated inverse $B_k$ of the Hessian.

% vim:sw=2:wrap:cc=100
