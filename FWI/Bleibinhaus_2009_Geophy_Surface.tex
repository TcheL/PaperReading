\renewcommand{\pmk}{Bleibinhaus\_2009\_Geophy\_Surface scattering in FWI}
\renewcommand{\prf}{FWI/\pmk.pdf}
\renewcommand{\pti}{Effects of surface scattering in full-waveform inversion}
\renewcommand{\pay}{Florian Bleibinhaus and St\'ephance Rondenay, 2009}
\renewcommand{\pjo}{Geophysics}
\renewcommand{\pda}{2019/5/6 Mon.}

\section{\pinfo}
\subsection{Introduction}
Resulting waveform models show artifacts and a loss of resolution
from neglecting the free suface in the inversion,
but the inversions are stable.
%, and they still improve the resolution of kinematic models.

\begin{enumerate}[\hspace{10mm}*]
  \item 2D, isotropic, acoustic or viscoacoustic, and FD frequency-domain
    methods: Hicks and Pratt, 2001; Operto \etal, 2004; Ravaut \etal, 2004;
    Operto \etal, 2006; Bleibinhaus \etal, 2007; Gao \etal, 2007;
    Malinowski and Operto, 2008.
  \item FWI study on a physical scale model: Pratt, 1999.
  \item \sline
  \item Invert elastic phases in the acoustic approximation:
    Barnes and Charara, 2008; Choi \etal, 2008.
  \item The impact of attenuation and the possibility of
    retrieving attenuation structure: Kamei and Pratt, 2008.
  \item \sline
  \item Compute the pressure field from the divergence of
    the particle velocity: Dougherty and Stephen, 1988.
  \item Compute the frequency-domain wavefields with
    the phase-sensitive detection method: Nihei and Li, 2007.
  \item \sline
  \item \mynnem{Gardner's formula} (from velocity to density):
    \myidxoo{Other}{Other}{Gardner's formula}
    Gardner \etal, 1974.
  \item \sline
  \item Viscoelastic finite-difference time-domain code:
    Robertsson \etal, 1994; Robertsson, 1996; Robertsson and Holliger, 1997.
  \item Image method for an irregular free surface: Levander, 1988.
  \item Viscoelastic 3D code: Bohlen and Saenger, 2006.
  \item \sline
  \item Travel time tomography using the eikonal solver: Hole, 1992.
  \item \sline
  \item Multiscale approach to mitigate the nonlinearities inherent to FWI:
    Bunks \etal, 1995; Pratt \etal, 1996.
  \item Viscoelastic frequency-domain code:
    Pratt and Worthington, 1990; Pratt \etal, 1998.
\end{enumerate}

\subsection{Test model}
The strong attenuation could mitigate the effects of surface-scattered waves.

\subsection{Starting model}
Wavelengths than can be resolved by full-waveform inversion are
closely related to the bandwidth of the data.
In particular, low frequencies are required to resolve
the long-wavelength structure of the model.

Typically, real applications derive starting models from traveltime tomography.

\subsection{Waveform inversion}
Real data amplitudes are too strongly affected by variations of
near surface attenuation and coupling conditions.

The amplitudes are sensitive only to the spatial gradient of the velocities,
not to the velocities themselves, and their resolving power is
relatively poor compared to the phase
(\myno{I did not find the conclusion from} Shin and Min, 2006).

\subsection{Conclusions}
Strong topography produces additional scattering,
and this scattering generally reduces the resolution.
However, strong topography also destroys the coherency of multiples and
mitigates reverberations, and the corresponding artifacts are reduced.

It is possible to mimic some effects of an irregular surface by
a weak contrast along a staircase function.

% vim:sw=2:wrap
