\renewcommand{\pmk}{Virieux\_2009\_Geophy\_Overview of FWI}
\renewcommand{\prf}{FWI/\pmk.pdf}
\renewcommand{\pti}{An overview of full-waveform inversion
in exploration geophysics}
\renewcommand{\pay}{J. Virieux and S. Operto, 2009}
\renewcommand{\pjo}{Geophysics}
\renewcommand{\pda}{2019/2/8 Fri.}

\section{\pinfo}

%%% === dividing line: 1.0 ===
\subsection{Introduction}
Key ingredients of FWI are an efficient forward-modeling engine
and a local differential approach,
in which the gradient and the Hessian operators are efficiently estimated.

\begin{enumerate}[\hspace{10mm}*]
  \item Main discoveries by using traveltime information:
    Oldham, 1906; Gutenberg, 1914; Lehmann, 1936.
  \item Differential seismograms estimated through the Born approximation:
    Gilbert and Dziewonski, 1975; Woodhouse and Dziewonski, 1984.
  \item \sline
  \item Exploding-reflector after some kinematic corrections and
    amplitide summation: Claerbout, 1971 \& 1976.
  \item \sline
  \item Two-step workflow - construct the macromodel, and then
    the amplitude projection: Claerbout and Doherty, 1972; Gazdag, 1978;
    Stolt, 1978; Baysal \etal, 1983; Yilmaz, 2001; Biondi and Symes, 2004.
  \item Various approaches for iterative updating of
    the macromodel reconstruction: Snieder \etal, 1989; Docherty \etal, 2003.
  \item \sline
  \item Forward-modeling: reflectivity techniques in layered media,
    Kormendi and Dietrich, 1991; finite-difference techniques,
    Kolb \etal, 1986 \& Ikelle \etal, 1988 \& Crase \etal, 1990 \&
    Pica \etal, 1990 \& Djikp\'ess\'e and Tarantola, 1999;
    finite-element methods, Choi \etal, 2008; extended ray theory,
    Cary and Chapman, 1988 \& Koren \etal, 1991 \&
    Sambridge and Drijkoningen, 1992.
  \item Generalized Radon reconstruction techniques: Beylkin, 1985;
    Bleistein, 1987; Beylkin and Burridge, 1990.
  \item Recast the asymptotic Radon transform as an iterative least-squares
    optimization after diagonalization the Hessian opeartor:
    theory, Jin \etal, 1992 \& Lambar\'e \etal, 1992;
    2D application, Thierry \etal, 1999b \& Operto \etal, 2000;
    3D extension, Thierry \etal, 1999a \& Operto \etal, 2003.
  \item \sline
  \item Volumetric methods: finite-element methods, Marfurt, 1984 \&
    Min \etal, 2003; finite-difference methods, Virieux, 1986;
    finite-volume methods, Brossier \etal, 2008;
    pseudospectral methods, Danecek and Seriani, 2008.
  \item Boundary integral methods: reflectivity methods, Kennett, 1983.
  \item Generalized screen methods: Wu, 2003.
  \item Discrete wavenumber methods: Bouchon \etal, 1989.
  \item Generalized ray methods: WKBJ; Maslov seismograms, Chapman, 1985.
  \item Full-wave theory: de Hoop, 1960.
  \item Diffraction theory: Klem-Musatov and Aizenbery, 1985.
  \item \sline
  \item Matrix notations to denote the partial-differential operators
    of the wave equation: Marfurt, 1984; Carcione \etal, 2002.
  \item Finite-difference method: Virieux, 1986; Levander, 1988;
    Graves, 1996; Operto \etal, 2007.
  \item Direct-solver approach for 2D forward problems:
    Jo \etal, 1996; Stekl and Pratt, 1998; Hustedt \etal, 2004.
  \item Iterative solvers for the time-harmonic wave equation:
    Riyanti \etal, 2006 \& 2007; Plessix, 2007;
    Erlangga and Herrmann, 2008; Saad, 2003 (Krylov subspace methods).
  \item Hybrid direct-iterative approach based on
    a domain decomposition method and the Schur complement system:
    Saad, 2003; Sourbier \etal, 2008.
  \item Time windowing to mitigate the nonlinearity of the inversion:
    Sears \etal, 2008; Brossier \etal, 2009a.
  \item More detailed complexity analyses of seismic modeling
    based on different numerical approaches:
    Plessix, 2007 \& 2009; Virieux \etal, 2009.
  \item Discussion on time-domain versus frequency-domain
    seismic modeling with application to FWI:
    Vigh and Starr, 2008b; Warner \etal, 2008.
  \item \sline
  \item Length method: Menke, 1984.
  \item Probabilistic maximum likelihood or generalized inverse formulations:
    Menke, 1984; Tarantola, 1987; Scales and Smith, 1994; Sen and Stoffa, 1995.
  \item Discussion on alternative parameterizations:
    Appendix A in Pratt \etal, 1998.
  \item Line search method: Gauthier \etal, 1986; Tarantola, 1987;
    Sambridge \etal, 1991 (extend to multiple-parameter classes
    using a subspace approach).
  \item The pseudo-Hessian, dividing the gradient by the diagonal terms of
    the Hessian: Shin \etal, 2001a.
  \item Conjugate gradient method in FWI: Mora, 1987;
    Tarantola, 1987; Crase \etal, 1990.
  \item Polak-Ribi\`ere formula: Polak and Ribi\`ere, 1969.
  \item The BFGS algorithm: Nocedal, 1980.
  \item Comparison between CG and L-BFGS for a realistic application of
    multiparameter FWI: Brossier \etal, 2009a.
  \item Gauss-Newton and Newton algorithms: Akcelik, 2002; Askan \etal, 2007;
    Askan and Bielak, 2008; Epanomeritakis \etal, 2008.
  \item Apply some regularizations to the inversion of FWI:
    Menke, 1984; Tarantola, 1987; Scales \etal, 1990.
  \item Bayesian formulation of FWI: Tarantola, 1987.
  \item Weighting by a power of the source-receiver offset to strengthen
    the contribution of large-offset data for crustal-scale imaging:
    Operto \etal, 2006.
  \item Review on regularization methods: Hansen, 1998.
  \item Multidimensional adaptive Gaussian smoother:
    Ravaut \etal, 2004; Operto \etal, 2006.
  \item Low-pass filter in the wavenumber domain: Sirgue, 2003.
  \item Implement total variation regularization as a multiplicative constraint
    in the original misfit function: van den Berg and Abubakar, 2001.
  \item The weighted $L_2$-norm regularization to frequency-domain FWI:
    Hu \etal, 2009; Abubakar \etal, 2009.
  \item A clear interpretation of the gradient and Hessian: Pratt \etal, 1998.
  \item Radiation patterns ofr the isotropic acoustic, elastic,
    and viscoelastic wave equations: Wu and Aki, 1985; Tarantola, 1986;
    Ribodetti and Virieux, 1996; Forgues and Lambar\'e, 1997.
  \item The adjoint-state method of the optimization theory:
    Lions, 1972; Chavent, 1974.
  \item The assimilation method in fluid mechanics:
    Talagrand nad Courtier, 1987.
  \item The adjoint-state method for seismic problems: Tromp \etal, 2005;
    Askan, 2006; Plessix, 2006; Epanomeritakis \etal, 2008.
  \item Compute the diagonal terms of the approximate Hessian
    for a decimated shot acquisition: Operto \etal, 2006.
  \item An approximation of the diagonal Hessian: Shin \etal, 2001a.
  \item The scattering-integral method based on the explicit building
    of the sensitivity matrix: Chen \etal, 2007.
  \item LSQR algorithm: Paige and Saunders, 1982a.
  \item A comparative complexity analysis of the adjoint approach and
    the scattering-integral approach: Chen \etal, 2007.
  \item The derivation in the frequency domain of the gradient of
    the misfit function in the matrix and functional formalisms:
    Gelis \etal, 2007.
  \item The differential semblance optimization: Pratt and Symes, 2002.
  \item Heuristic criteria to stop the iteration of the inversion:
    Jaiswal \etal, 2009.
  \item Source-independent misfit functions: Lee and Kim, 2003;
    Zhou and Greenhalgh, 2003.
  \item \sline
  \item Generalized diffraction tomography: Devaney and Zhang, 1991;
    Gelius \etal, 1991.
  \item Diffraction tomographyc: Devaney, 1982; Wu and Toksoz, 1987;
    Sirgue and Pratt, 2004; Lecomte \etal, 2005.
  \item Generalized Radon transform: Miller \etal, 1987.
  \item Ray + Born migration/inversion: Lambar\'e \etal, 2003.
  \item Decimate the wavenumber-converage redundancy in frequency-domain FWI
    by limiting the inversion to a few discrete frequencies:
    Pratt and Worthington, 1990; Sirgue and Pratt, 2004;
    Brenders and Pratt, 2007a.
  \item A guideline for selecting the frequencies for
    the frequency-domain FWI: Sirgue and Pratt, 2004.
  \item \mynem{Wavepath}: a frequency-domain sensitivity kernel
    for point sources: Woodward, 1992.


























\end{enumerate}

The limited offsets recorded by seismic reflection surveys and
the limited-frequency bandwidth of seismic sources make seismic imaging
poorly sensitive to intermediate wavelengths.
In complex geologic environments, building an accurate velocity
background model for migration is challenging.

The gradient of the misfit function
along which the perturbation model is searched
can be built by crosscorrelating the incident wavefield
emitted from the source and
the back-propagated residual wavefields.
The perturbation model obtained after the first iteration
looks like a migrated image obtained by reverse-time migration.
Difference: the seismic wavefield recorded at the receiver
is back-propagated in reverse-time migration,
whereas the data misfit is back-propagated in the scheme.

%%% === dividing line: 2.0 ===
\subsection{The Forward Problem}
\myno{Explicit time-marching algorithm}: The value of the wavefield at a time step
at a spatial position is inferred from the value of the wavefields
at previous time steps.

In the time domain,
\[ \mbf M(\mbf x)\frac{d^2\mbf u(\mbf x,t)}{dt^2}=
\mbf A(\mbf x)\mbf u(\mbf x,t)+\mbf s(\mbf x,t) \]
where $\mbf M$: the mass matrix; $\mbf A$: the stiffness matrix.

In the frequency domain,
\[ \mbf B(\mbf x,\omega)\mbf u(\mbf x,\omega)=\mbf s(\mbf x,\omega) \]
where the impedance matrix $\mbf B$ has a symmetirc pattern
but is not symmetirc.
Once the decomposition is performed in the direct-solver approach,
the equation is efficiently solved for multiple sources
using forward and backward substitutions.

The spatial reciprocity of Green's functions can be exploited in FWI
to mitigate the number of forward problems if the number of receivers
is significantly smaller than the number of sources.
Of note, the spatial reciprocity is satisfied theoretically for
the unidirectional sensor and the unidirectional impulse soruce,
but also can be used for explosive sources.

%%% === dividing line: 3.0 ===
\subsection{Least-squares local optimization}
The misfit vector:
\[ \Delta\mbf d=\mbf d_{obs}-\mbf d_{cal}(\mbf m) \]
\[ \mbf d_{cal}=\mycR\mbf u \]
where $\mycR$: the detection operator.

%% --- dividing line: 3.1 ---
\subsubsection{Born approximation and linearization}
The least-squares norm:
\[ C(\mbf m)=\frac{1}{2}\Delta\mbf d^\dagger\Delta\mbf d \]
where $\dagger$: the adjoint operator (\mynno{transpose conjugate}).
In the vicinity of $\mbf m_0$ ($\mbf m=\mbf m_0+\Delta\mbf m$):
\[ C(\mbf m_0+\Delta\mbf m)=C(\mbf m_0)
  +\sum_{j=1}^M\frac{\partial C(\mbf m_0)}{\partial m_j}\Delta m_j
  +\frac{1}{2}\sum_{j=1}^M\sum_{k=1}^M\frac{\partial^2C(\mbf m_0)}{\partial m_j\partial m_k}\Delta m_j\Delta m_k
  +\mathcal O(\mbf m^3) \]
\[ \frac{\partial C(\mbf m)}{\partial m_l}=\frac{\partial C(\mbf m_0)}{\partial m_l}
  +\sum_{j=1}^M\frac{\partial^2C(\mbf m_0)}{\partial m_j\partial m_l}\Delta m_j \]
When the first derivative vanishes, the perturbation model:
\[ \Delta\mbf m=-\Big[\frac{\partial^2C(\mbf m_0)}{\partial\mbf m^2}\Big]^{-1}
  \frac{\partial C(\mbf m_0)}{\partial\mbf m)} \]

The above equation gives the minimum of the misfit function in one iteration
in the case of linear forward.
Because of the nolinear relationship between the data and the model in FWI,
the inversion needs to be iterated several times.

%% --- dividing line: 3.2 ---
\subsubsection{Normal equations}
The derivative:
\begin{align*}
  \frac{\partial C(\mbf m)}{\partial m_l} & =-\frac{1}{2}\sum_{i=1}^N
      \Big[\Big(\frac{\partial d_{cal_i}}{\partial m_l}\Big)(d_{obs_i}-d_{cal_i})^*
      +(d_{obs_i}-d_{cal_i})\frac{\partial d_{cal_i}^*}{\partial m_l}\Big] \\
    & =-\sum_{i=1}^N\myRe\Big[\Big(\frac{\partial d_{cal_i}}{\partial m_l}\Big)^*
      (d_{obs_i}-d_{cal_i})\Big]
      \myno{=-\sum_{i=1}^N\myRe\Big[\Big(\frac{\partial d_{cal_i}}{\partial m_l}\Big)
      (d_{obs_i}-d_{cal_i})^*\Big]} \\
  \nabla C_{\mbf m} & =\frac{\partial C(\mbf m)}{\partial\mbf m}
      =-\myRe\Big[\Big(\frac{\partial\mbf d_{cal}(\mbf m)}{\partial\mbf m}\Big)^*
      \big(\mbf d_{obs}-\mbf d_{cal}(\mbf m)\big)\Big] \\
    & =-\myRe[\mbf J^\dagger\Delta\mbf d] \myno{=-\myRe[\mbf J^T\Delta\mbf d^*]}
\end{align*}
where $\myRe$ and $*$: the real part and the conjugate, respectively;
$\mbf J$: the sensitivity or the \Frechet derivative matrix.

The Hessian:
\[ \frac{\partial^2 C(\mbf m_0)}{\partial\mbf m^2}=\myRe[\mbf J_0^\dagger\mbf J_0]
  +\myRe\Big[\frac{\partial\mbf J_0^{\myde t\dagger}}{\partial\mbf m^{\myde t}}
  (\Delta\mbf d_0^{\myde *}~\Delta\mbf d_0^{\myde *}~\cdots~\Delta\mbf d_0^{\myde *})\Big] \]
and the normal equation:
\[ \Delta\mbf m=-\Big\{\myRe\Big[\mbf J_0^\dagger\mbf J_0
  +\frac{\partial\mbf J_0^{\myde t\dagger}}{\partial\mbf m^{\myde t}}
  (\Delta\mbf d_0^{\myde *}~\Delta\mbf d_0^{\myde *}~\cdots~\Delta\mbf d_0^{\myde *})
  \Big]\Big\}^{-1}\myRe[\mbf J_0^t\Delta\mbf d_0] \]

% ... dividing line: 3.2 (1) ...
\paragraph{The conjugate gradient method}
The direction of updating model:
\[ \mbf p^{(n)}=\nabla C^{(n)}+\beta^{(n)}\mbf p^{(n-1)} \]
\[ \beta^{(n)}=\frac{(\nabla C^{(n)}-\nabla C^{(n-1)})^t\nabla C^{(n)}}{||\nabla C^{(n)}||^2} \]
In FWI, the preconditioned gradient
$W^{-1}\nabla C^{(n)}$ is used for $\mbf p^{(n)}$,
and $W$: the weighting operator.

% ... dividing line: 3.2 (2) ...
\paragraph{Quasi-Newton algorithms}
The L-BFGS algorithm needs a negligible storage and computational cost
compared to the conjugate gradient algorithm.
For multiparameter FWI, the L-BFGS algorithm provides a suitable scaling of
the gardients computed for each parameter class.

%% --- dividing line: 3.3 ---
\subsubsection{Regularization and precoditioning}
Augment the misfit:
\[ \mathcal{C}(\mbf m)=\frac{1}{2}\Delta\mbf d^\dagger\mbf W_d\Delta\mbf d
  +\frac{1}{2}\varepsilon(\mbf m-\mbf m_{prior})^\dagger\mbf W_m(\mbf m-\mbf m_{prior}) \]
where $\mbf W_d$: the data-weighting operator;
$\mbf W_m$: the model-roughness operator.

For linear problems,
\[ \Delta\mbf m=-\{\myRe(\mbf J_0^\dagger\mbf W_d\mbf J_0)+\varepsilon\mbf W_m\}^{-1}
  \myRe[\mbf J_0^\dagger\mbf W_d\Delta\mbf d_0] \]
\[ \Delta\mbf m=-\mbf W_m^{-1}\{\myRe(\mbf J_0\mbf W_m^{-1}\mbf J_0^\dagger)
  +\varepsilon\mbf W_d^{-1}\}^{-1}\myRe[\mbf J_0^\dagger\Delta\mbf d_0] \]
where $\mbf W_m^{-1}$: the smoothing operator.

For the steepest-descent algorithm,
\[ \Delta\mbf m=-\alpha\mbf W_m^{-1}\myRe[\mbf J_0^\dagger\mbf W_d\Delta\mbf d_0] \]

%% --- dividing line: 3.4 ---
\subsubsection{The gradient and Hessian}
The gradient is formed by the zero-lag correlation between
the partial-derivative wavefield and the data residual.
It represents perturbation wavefields scattered by the missing heterogeneities
in the starting model.

The approximate Hessian is formed by the zero-lag correlation between
the partial-derivative wavefields.
Scaling the gradient by the diagonal terms of the approximate Hessian
removes from the gradient the geometric amplitude of the partial-derivative
wavefields and the residuals.

The back propagation in time is indicated by the conjugate operator
in the frequency domain.

The underlying imaging principle is reverse-time migration, which relies on
the correspondence of the arrival times of the incident wavefield and
the back-propagated wavefield at the position of heterogeneity.

The scattering-integral approach outperforms the adjoint approach
for a regional tomographic problem (Chen \etal, 2007).
But the superiority is dependent on the acquisition geometry and
the number of model parameters.

%% --- dividing line: 3.5 ---
\subsubsection{Source estimation}
The solution for the source is given by the expression:
\[ \mbf s=\frac{\mbf g_{cal}(\mbf m_0) \mbf d_{obs}^t}
  {\mbf g_{cal}(\mbf m_0) \mbf g_{cal}(\mbf m_0)^t} \]
where $\mbf g_{cal}(\mbf m_0)$ the Green's functions with
the starting model $\mbf m_0$.

The source and the medium are updated alternatively
over iterations of the FWI.

To normalize each seismogram of a shot gather by the sum of
all the seismograms removes the dependency of data with respect
to the source (Lee and Kim, 2003; Zhou and Greenhalgh, 2003).

%%% === dividing line: 4.0 ===
\subsection{Some key features}

%% --- dividing line: 4.1 ---
\subsubsection{Resolution power of FWI}
For plane wave propagating in a homogeneous background model,
if no amplitude effects, the gradient has the form of
a truncated Fourier series:
\[ \nabla C(\mbf m) = -\omega^2\sum_\omega\sum_s\sum_r \myRe\{
  e^{-ik_0(\hat{\mbf s}+\hat{\mbf r})\cdot\mbf x} \Delta\mbf d\} \]

The relationship between the experimental setup and
the spatial resolution of the reconstruction:
\[ \mbf k=\frac{2f}{c_0}\cos\Big(\frac{\theta}{2}\Big)\mbf n \]
where $\mbf n$: the unit vector of the slowness.

Some conclusions:
Frequency and aperture have redundant control of the wavenumber coverage.
The low frequencies of the data and the wide apertures help resolve
the intermediate and large wavelengths of the medium;
The highest frequency leads to a maximum resolution of half a wavelength
if normal-incidence reflections are recorded.

The width of the first Fresnel zone is $\sqrt{\lambda L}$,
where $L$: the source-receiver offset.





























%% --- dividing line: 4.2 ---
\subsubsection{Multiscale FWI}

%% --- dividing line: 4.3 ---
\subsubsection{Parallel implementation of FWI}

%% --- dividing line: 4.4 ---
\subsubsection{Variants of classic FWI}

%% --- dividing line: 4.5 ---
\subsubsection{Starting models for FWI}




% vim:sw=2:wrap
