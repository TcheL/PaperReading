\renewcommand{\pmk}{Moghaddam\_2013\_Geophy\_Stochastic gradient method}
\renewcommand{\prf}{FWI/\pmk.pdf}
\renewcommand{\pti}{A new optimizatioin approach for source-encoding full-waveform inversion}
\renewcommand{\pay}{Peyman P. Moghaddam, Henk Keers, Felix J. Herrmann, \etal, 2013}
\renewcommand{\pjo}{Geophysics}
\renewcommand{\pda}{2017/6/11 Sun.}

\section{\pinfo}
\subsection{Introduction}
\begin{enumerate}[\hspace{10mm}*]
  \item FWI on a large scale: Virieux and Operto, 2009; Kapoor \etal, 2010; Vigh \etal, 2010.
  \item Conventional FWI: Tarantola, 1984 \& 1986; Mora, 1987; Crase \etal, 1990; ...
  \item Source encoding technique: Krebs \etal 2009; Li and Herrmann, 2010;
    Moghaddam and Herrmann, 2010; van Leeuwen \etal, 2011; Haber \etal, 2012; Li \etal, 2012.
  \item \sline
  \item Stochastic optimization method: Goldberg, 1989; Spall, 1992.
  \item \sline
  \item Marmousi model: Bourgeois \etal, 1991.
  \item \sline
  \item \mynem{The adjoint-state method} to avoid the computation of sensitivity matrix:
    Lions and Magenes, 1972; Lailly, 1983; Tarantola, 1984; Giles \etal, 2003;
    Plessix, 2006; Virieux and Operto, 2009.
  \item The limited-memory Broyden-Fletcher-Goldfarb-Shanno (\myem{LBFGS}) method:
    Byrd \etal, 1995; Mulder and Plessix, 2004; Nocedal and Wright, 2006; Plessix, 2004.
  \item Preconditioned conjugate gradient method: Ravaut \etal, 2004.
  \item Gauss-Newton method: Virieux and Operto, 2009.
  \item The online LBFGS (oLBFGS): Schraudolph \etal, 2007; Yu \etal, 2010.
  \item Stochastic gradient descent: Schraudolph \etal, 2007; Sunehag \etal, 2009.
\end{enumerate}

The misfit function, and therefore also its gradient, for source-encoding waveform inversion
is an unbiased random estimation of the misfit function used in conventional waveform inversion.

Main drawbacks of FWI: the requirement to have an accurate initial model;
and expensive computational cost.

Source encoding uses a linear combinations of all shots, with random weights assigned to each shot.

\subsection{Stochastic optimization}
\paragraph{Stochastic gradient descent}
\myidx{Inversion}{Iteration}{stochastic gradient descent}
Stochastic gradient descent is:
\[ \sigma_{k+1}=\sigma_k-\eta_k\nabla J(\sigma_k,\mbf w_k) \]
where $k$ the iteration number, $\eta_k$ the step length, $J$ the misfit function,
$\sigma_k$ the model at iteration $k$ and \mynem{$\mbf w_k$ the current randomized weight}.

\paragraph{Stochastic LBFGS}
\myidx{Inversion}{Iteration}{stochastic LBFGS}
Each step of the LBFGS algorithm takes:
\[ \sigma_{k+1}=\sigma_k-\eta_k\mbf H_k\nabla J(\sigma_k,\mbf w_k) \]
where the inverse Hessian matrix $\mbf H_k$ updated in each iteration
by (refer to the last second formula of
\href{https://en.wikipedia.org/wiki/Broyden-Fletcher-Goldfarb-Shanno_algorithm}{Wikipedia page}):
\[ \mbf H_{k+1}=\mbf V_k^T\mbf H_k\mbf V_k+\rho_k\mbf s_k\mbf s_k^T \]
with $\rho_k=\nicefrac{1}{\mbf y_k^T\mbf s_k}$, $\mbf V_k=\mbf I-\rho_k\mbf y_k\mbf s_k^T$
and $\mbf s_k=\sigma_{k+1}-\sigma_k$,
$\mbf y_k=\nabla J(\sigma_{k+1},\mbf w_k)-\nabla J(\sigma_k,\mbf w_k)$.
Note that for construction of $\mbf y_k$, take the same random weighting $\mbf w_k$
for the current gradient at $k+1$ and the previous one at $k$.

The LBFGS routine is carried out in two steps.
First, the latest $m$ iterations are calculated.
Second, the routine updates the LBFGS direction as the following:
\myidx{Other}{Algorithm}{LBFGS}
\begin{enumerate}[\hspace{15mm}1:~]
%\begin{enumitem}%[\hspace{15mm}1:~]
  \item $\mbf q\leftarrow\nabla J(\mbf m_k,\mbf w_k)$
  \item $\mbf H_k^0\leftarrow\nicefrac{(\mbf y_k^T\mbf s_k)}{(\mbf y_k^T\mbf y_k)}$
  \item FOR $i=k$ to $k-m+1$
  \item \quad\quad $\alpha_i\leftarrow\rho_i\mbf s_i^T\mbf q$
  \item \quad\quad $\mbf q\leftarrow\mbf q-\alpha_i\mbf y_i$
  \item END FOR
  \item $\mbf r\leftarrow\mbf H_k^0\mbf q$
  \item FOR $i=k-m+1$ to $k$
  \item \quad\quad $\beta\leftarrow\rho_i\mbf y_i^T\mbf r$
  \item \quad\quad $\mbf r\leftarrow\mbf r+\mbf s_i(\alpha_i-\beta)$
  \item END FOR
  \item Stop with $\mbf r=\mbf H_{k+1}\nabla J(\mbf m_{k+1},\mbf w_{k+1})$
\end{enumerate}
%\end{enumitem}

\paragraph{Stochastic oLBFGS}
\myidx{Inversion}{Iteration}{stochastic oLBFGS}
For better convergence, the oLBFGS method uses
$\mbf y_k=\nabla J(\mbf m_{k+1},\mbf w_k)-\nabla J(\mbf m_k,\mbf w_k)+\lambda\mbf s_k$.
And the step $\mbf r\leftarrow\mbf H_k^0\mbf q$ in the above procedures is replaced by:
\[ \mbf r=\frac{\mbf q}{\min(k,m)}\sum_{i=1}^{\min(k,m)}\frac{\mbf s_{k-i}^T\mbf y_{k-i}}{\mbf y_{k-i}^T\mbf y_{k-i}} \]
where we can set $\lambda=0.1\cdot\nicefrac{||J(\mbf m_0,\mbf w_0)||_2^2}{||\mbf m_0||_2^2}$.

\paragraph{Integrated stochastic gradient descent}
\myidx{Inversion}{Iteration}{integrated stochastic gradient descent}
To accelerate the convergence, in the integrated stochastic gradient descent (iSGD) method,
the iteration step takes:
\[ \sigma_{k+1}=\sigma_k-\eta_k\overline{\nabla J(\sigma_k)} \]
\[ \overline{\nabla J(\sigma_k)}=\frac{\sum_{i=k-m}^ke^{\alpha(i-k)}\nabla J(\sigma_i,\mbf w_i)}{\sum_{i=k-m}^ke^{\alpha(i-k)}} \]
where we can set $m=10$.

% vim:sw=2:wrap
