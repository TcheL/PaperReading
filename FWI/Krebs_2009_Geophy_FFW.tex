\renewcommand{\pmk}{Krebs\_2009\_Geophy\_FFW using encoded sources}
\renewcommand{\prf}{FWI/\pmk.pdf}
\renewcommand{\pti}{Fast full-wavefield seismic inversion
using encoded sources}
\renewcommand{\pay}{Jerome R. Krebs, John E. Anderson, David Hinkley
\etal, 2009}
\renewcommand{\pjo}{Geophysics}
\renewcommand{\pda}{2019/3/28 Thu.}

\section{\pinfo}

\subsection{Intruoduction}
The encoding step forms a single gather from many input source gathers.

\begin{enumerate}[\hspace{10mm}*]
  \item Iterative gradient search methods: Nocedal and Wright, 2006.
  \item \sline
  \item Frequency-domain direct-solver technique: Marfurt, 1984;
    Pratt and Worthington, 1990.
  \item \sline
  \item Explicit time-domain simulator: Tarantola, 1987.
  \item Iterative solver-based frequency-domain simulator: Erlanga \etal, 2006;
    Operto \etal, 2006; Riyanti \etal, 2006.
  \item \sline
  \item Inverting only a few frequencies: Pratt, 1999; Sirgue and Pratt, 2004.
  \item Inverting coherent sums of sources: Berkhout, 1992; Warner \etal, 2008.
  \item Inverting sums of widely spaced sources: Mora, 1987;
    Capdeville \etal, 2005.
  \item \sline
  \item Incoherent source sums:
    in seismic data acquisition, Neelamani and Krohn, 2008;
    in wave-equation migration, Romero \etal, 2000;
    in seismic simulation, Ikelle, 2007 \& Neelamani \etal, 2008.
  \item \sline
  \item Prefectly matched layer boundary conditions:
    Marcinkovich and Olsen, 2003.
  \item Random phase encoding: Romero \etal, 2000.
  \item The Hestenes-Stiefel conjugate gradient algorithm:
    Nocedal and Wright, 2006.
  \item Multiscale inversion: Bunks \etal, 1995.
  \item \sline
  \item \mynnem{Marmousi II model}
    \myidxox{Other}{Model}{Marmousi II: 2-D elastic}:
    Martin, 2004 (please click \href{http://www.agl.uh.edu/downloads/downloads.htm}{here}
    to download the model data).
\end{enumerate}

FWI attempts to find an earth model that best explains
the measured seismic data and also satisfies known constrains.

Popular encoding methods include phase reversal, phase shifting, time shifting,
and convolution with random sequences.
Most methods that exploit incoherent source sums suffer from
large amounts of crosstalk noise.

Altering the random-number seed used to generate the source-encoding functions
between iterations can achieve large efficiency gains for FWI
without significant crosstalk noise.

\subsection{Theory}
For the encoded simultaneous-source FWI (ESSFWI\myidx{Inversion}{FWI}{ESSFWI}),
the objective function:
\[ h(u(c),c) = \Bigg| u\Bigg(c,\sum_{n=1}^{N_s}e_n\otimes s_n\Bigg)
  - \sum_{n=1}^{N_s}e_n\otimes d_n\Bigg|^2 \]
where $e_n$: the encoding sequence,
and $\otimes$: convolution with respect to time.
In general, $e_n\neq e_m$ for $n\neq m$.

An incoherently encoded gather illuminates more of the model
than a point-source gather,
and has a much broder spectrum of wave-propagation directions
than a coherently encoded gather.

\subsection{Methods}
A normalized random phase code with only one sample
(\mynem{randomly multiplying the shot gathers by $+1$ or $-1$})
gives the best convergence rate and the most efficient inversion.

\subsection{Test}
If multiscale techniques (Bunks \etal, 1995) are used
in FWI to avoid local minima,
the measured data must have high S/N at very low frequencies or
the initial model must accurately predict seismic traveltimes.

\subsection{Conclusions}
ESSFWI is significantly more sensitive to ambient noise levels than is FWI,
so we must be careful to limite the number of sources encoded into
a simultaneous-source gather if ambient noise levels are high.

ESSFWI efficiency gains are relatively insensitive to
the accuracy of the starting model.

Single-sample codes work as well as longer, more orthogonal codes.

% vim:sw=2:wrap
