\renewcommand{\pmk}{Yuan\_2016\_GJI\_Double-difference adjoint tomography}
\renewcommand{\prf}{FWI/\pmk.pdf}
\renewcommand{\pti}{Double-difference adjoint seismic tomography}
\renewcommand{\pay}{Yanhua O. Yuan, Frederik J. Simons and Jeroen Tromp, 2016}
\renewcommand{\pjo}{Geophys. J. Int.}
\renewcommand{\pda}{2019/8/25 Sun.}

\section{\pinfo}

\subsection{Introduction}
Differential measurements between stations reduce the influence of
the source signature and systematic errors.

\begin{enumerate}[\hspace{10mm}*]
  \item Jointly update source terms and structural model parameters:
    Pavlis and Booker, 1980; Spencer and Gubbins, 1980; Abers and Roecker, 1991;
    Thurber, 1992; Widiyantoro \etal, 2000; Panning and Romanowicz, 2006;
    Tian \etal, 2011; Simmons \etal, 2012.
  \item \sline
  \item The fundamental ideas of making differential measurements:
    Brune and Dorman, 1963; Passier and Snieder, 1995.
  \item Double-differnece inversion in earthquake location:
    Poupinet \etal, 1984 \& Got \etal, 1994; \mynnem{hypoDD}
    \myidxox{Other}{Software}{hypoDD: double-difference event location},
    Waldhauser and Ellsworth, 2000.
  \item The double-difference tomography code \mynnem{tomoDD}
    \myidxox{Other}{Software}{tomoDD: double-difference tomography}:
    Zhang and Thurber, 2003.
  \item Differencing differentail measurements between pairs of stations:
    Monteiller \etal, 2005; Fang and Zhang, 2014.
  \item The double-difference technique in earthquake and source studies:
    Rubin \etal, 1999; Rietbrock and Waldhauser, 2004;
    Schaff and Richards, 2004.
  \item \sline
  \item \Frechet kernel for differential traveltimes:
    Dahlen \etal, 2000; Hung \etal, 2000.
  \item Measure the sensitivities of relative time delays between
    two nearby stations: Hung \etal, 2004.
  \item The elastic adjoint method in global seismology: Tromp \etal, 2005.
  \item \sline
  \item Re- or pre-condition to avoid overemphasizing areas with high-density
    ray path coverage: Curtis and Snieder, 1997; Spakman and Bijwaard, 2001;
    Fichtner and Trampert, 2011; Luo \etal, 2015.
  \item \sline
  \item Calculate relative times from a catalogue of absolute arrival times:
    VanDecar and Crosson, 1990.
  \item Calculate relative traveltime delays:
    from waveform cross-correlation analysis, Luo and Schuster, 1991;
    from the cross-correlation of envelopes, Yuan \etal, 2015.
  \item \sline
  \item Membrane surface wave: Tanimoto, 1990; Peter \etal, 2007.
  \item \mynnem{SPECFEM2D}
    \myidxox{Other}{Software}{SPECFEM2D}:
    Komatitsch and Vilotte, 1998.
  \item \mynnem{Global phase velocity map} at periods between 40 and 150 s:
    \myidxox{Other}{Model}{Global phase velocity map}
    Trampert and Woodhouse, 1995.
  \item \mynnem{seisDD}
    \myidxox{Other}{Software}{seisDD: double-difference adjoint tomography}:
    \refp{this paper} (please click \href{https://github.com/yanhuay/seisDD}{here}
    to download the code).
\end{enumerate}

While considering pairs of earthquakes recorded at each station to reduce
the effects of structural uncertainty on the source locations,
differencing differential measurements between pairs of stations recording
the same earthquake lessens the effect of uncertainties in the source terms
on the determination of Earth structure.

Types of seismic `measurement': absolute, relative, and differential.

By the partial cancellation of common sensitivities, double-difference
tomography illuminates areas of the model domain where ray paths are not
densely overlapping.

\subsection{The classical approach}
The cross-correlation traveltime difference between synthetic signals $s_i(t)$
and observation $d_i(t)$ over a window of length $T$:
\[ \Delta t_i = \mathop{\arg\max}_\tau \int_0^T s_i(t + \tau) d_i(t) dt \]
The objective function:
\[ \chi_\text{cc} = \frac{1}{2} \sum_i [\Delta t_i]^2 \]
The traveltime perturbation:
\[ \delta \Delta t_i = \frac{ \int_0^T \partial_t s_i(t) \delta s_i(t) dt }
  { \int_0^T \partial_t^2 s_i(t) s_i(t) dt }
  \myno{ = - \frac{ \int_0^T \partial s_i(t) \delta s_i(t) dt }
  { \int_0^T [ \partial_t s_i(t) ]^2 dt } } \]
The adjoint source:
\[ f_i^\dagger(\mbf x, t) = \Delta t_i \frac{ \partial_t s_i(T - t) }
  { \int_0^T \partial_t^2 s_i(t) s_i(t) dt } \delta(\mbf x - \mbf x_i) \]

\subsection{The double-difference way}
The differential cross-correlation traveltimes, between a pair of stations
indexed $i$ and $j$, are:
\[ \Delta t_{ij}^\text{syn} = \mathop{\arg\max}_\tau
  \int_0^T s_i(t + \tau) s_j(t) dt \]
\[ \Delta t_{ij}^\text{obs} = \mathop{\arg\max}_\tau
  \int_0^T d_i(t + \tau) d_j(t) dt \]
and the double-difference traveltime measurement is:
\[ \Delta\Delta t_{ij} = \Delta t_{ij}^\text{syn} - \Delta t_{ij}^\text{obs} \]

The misfit function:
\[ \chi_\text{cc}^\text{dd} = \frac{1}{2} \sum_i \sum_{j>i}
  [ \Delta\Delta t_{ij} ]^2 \]
and the derivative of the function:
\begin{align*}
  \delta\chi_\text{cc}^\text{dd} & = \sum_i \sum_{j>i} [ \Delta\Delta t_{ij} ]
    \delta\Delta t_{ij}^\text{syn} \\
  & = \int_0^T \bigg\{ \sum_i \bigg[ \sum_{j>i} \frac{\Delta\Delta t_{ij}}
    {N_{ij}} \partial_t s_j(t - \Delta t_{ij}^\text{syn}) \bigg] \delta s_i(t)
    - \sum_j \bigg[ \sum_{i<j} \frac{\Delta\Delta t_{ij}}{N_{ij}} \partial_t s_i
    (t + \Delta t_{ij}^\text{syn}) \bigg] \delta s_j(t) \bigg\} dt \\
  & \myno{ = \int_0^T \bigg\{ \sum_k \bigg[ \sum_{j>k} \frac{\Delta\Delta t_{kj}}
    {N_{kj}} \partial_t s_j(t - \Delta t_{kj}^\text{syn}) - \sum_{i<k}
    \frac{\Delta\Delta t_{ik}}{N_{ik}} \partial_t s_i(t
    + \Delta t_{ik}^\text{syn}) \bigg] \delta s_k(t) \bigg\} dt } \\
  & \myno{ = \int_0^T \bigg\{ \sum_k \bigg[ \sum_{i \neq k}
    \frac{\Delta\Delta t_{ki}}{N_{ki}} \partial_t s_i(t
    - \Delta t_{ki}^\text{syn}) \bigg] \delta s_k(t) \bigg\} dt
    = \int_0^T \bigg\{ \sum_k \bigg[ \sum_i \frac{\Delta\Delta t_{ki}}{N_{ki}}
    \partial_t s_i(t - \Delta t_{ki}^\text{syn}) \bigg] \delta s_k(t)
    \bigg\} dt }
\end{align*}
where
\[ N_{ij} = \int_0^T \partial_t^2 s_i(t + \Delta t_{ij}^\text{syn}) s_j(t) dt
  \myno{ = - \int_0^T \partial_t s_i(t + \Delta t_{ij}^\text{syn})
  \partial_t s_j(t) dt } \]
Thus, the corresponding adjoint sources are:
\[ f_i^\dagger(\mbf x, t) = + \sum_{j>i} \frac{\Delta\Delta t_{ij}}{N_{ij}}
  \partial_t s_j(T - [t - \Delta t_{ij}^\text{syn}])
  \delta(\mbf x - \mbf x_i) \]
\[ f_j^\dagger(\mbf x, t) = - \sum_{i<j} \frac{\Delta\Delta t_{ij}}{N_{ij}}
  \partial_t s_i(T - [t + \Delta t_{ij}^\text{syn}])
  \delta(\mbf x - \mbf x_j) \]
\[ \myno{ f_k^\dagger(\mbf x, t) = \sum_i \frac{\Delta\Delta t_{ki}}{N_{ki}}
  \partial_t s_i(T - [t - \Delta t_{ki}^\text{syn}])
  \delta(\mbf x - \mbf x_k) } \]

\subsection{Numerical experiments}
The double-difference approach provides powerful interstation constrains
on seismic structure, and is less sensitive to error or uncertainty
in the source.

The double-difference technique is ideal for the high-resolution investigation
of well-instrumented areas with limited natural seismic activity.

% vim:sw=2:wrap
