\renewcommand{\pmk}{Kennett\_2016\_GGG\_Multiscale seismic heterogeneity}
\renewcommand{\prf}{Others/\pmk.pdf}
\renewcommand{\pti}{Multiscale seismic heterogeneity in
the continental lithosphere}
\renewcommand{\pay}{B. L. N. Kennett and T. Furumura, 2016}
\renewcommand{\pjo}{Geochem. Geophys. Geosyst.}
\renewcommand{\pda}{2019/6/2 Sun.}

\section{\pinfo}
\subsection{Introduction}
Fine-scale heterogeneity is pervasive, but strongest in the crust.

\begin{enumerate}[\hspace{10mm}*]
  \item Summary of the properties of the continental lithosphere: Fowler, 2005.
  \item \sline
  \item A significant boundary in the lithosphere in cratonic regions is
    the $8^\circ$ discontinuity at about $90~km$ depth: Thybo and Perchuc, 1997.
  \item \sline
  \item A model with strong quasi-laminar structure in the top $100~km$ of
    the lithospheric mantle: Tittgemeyer \etal, 1996.
  \item Crustal scattering: Nielsen \etal, 2003.
  \item \sline
  \item A complex structure through the full lithosphere
    for the profile QUARTZ: Morozova \etal, 1999.
  \item \sline
  \item \mynnem{AuSREM},
    \myidxox{Other}{Model}{AuSREM: Australian lithosphere}
    the 3-D structure with node points at $0.5^\circ\times 0.5^\circ$
    at $5~km$ depth intervals down to $50~km$,
    with $25~km$ depth intervals to $300~km$:
    Kennett and Salmon, 2012 (Crustal component: Salmon \etal, 2013a;
    Mantle component: Kennett \etal, 2013; Moho: Salmon \etal, 2013b).
  \item The von K\'arm\'an distribution: Ishimaru, 1987.
  \item Radially anisotropic 3-D shear wave structure of the Australian
    lithosphere and asthenosphere: Yoshizawa, 2014.
  \item \sline
  \item A fourth-order staggered-grid scheme in space and second-order scheme
    in time finite difference simulation: Furumura and Chen, 2004.
  \item 3-D finite difference viscoelastic wave modelling: Hestholm, 1999.
  \item A description of the components of the seismic wavefield for regional
    to far-regional distances in terms of an operator representation:
    Kennett, 1989.
  \item The upper mantle low velocity zone: Thybo, 2008.
\end{enumerate}

Fine-scale structure is superimposed on the major changes in seismic wave speed,
and leads to localized impedance contrasts.

Body-wave tomographic results can help to refine structure further,
with potential horizontal resolution limited by station spacing,
but vertical smearing due to the relatively narrow cone of
incoming rays limits resolution in the upper mantle.

The extended high-frequency coda requires some form of
distributed heterogeneity through the lithosphere.

\subsection{Representing heterogeneity}
The von K\'arm\'an distribution with correlation lengths
$a_x$ in the horizontal and $a_z$ in the vertical direction:
\[ P(p, q) = \frac{4\pi\kappa\epsilon^2 a_x a_z}
  {(1 + \omega^2 a_x^2 p^2 + \omega^2 a_z^2 q^2)^{\kappa + 1}} \]
where $p$ and $q$: the horizontal and vertical slowness;
$\epsilon$: the RMS amplitude of wave speed deviation from the reference;
and $\kappa$: the Hurst exponent that specifies the rate of decrease
of short wavelengths.

The shallower bound on the lithosphere-asthenosphere transition (LAT)
is determined from the peak negative gradient of S wave speed,
and the deeper bound from the absolute minimun of S wave speed.

The LAT zone has low attenuation beneath the cratonic zones.

\subsection{Lithospheric heterogeneity}
Earth flattening is applied to the P and S wave speeds in order to include
the effect of the sphericity of the Earth using
a conventional rectangular-grid finite difference method.

Below the highest absolute wave speeds in the lithosphere,
the zone of diminishing wave speed leading into the asthenosphere,
has strong short-range heterogeneity with a squat aspect ratio.

The presence of strong crustal heterogeneity breaks up the conversions
from Pn to S, and scattering also has the effect of
coupling S waves with P in the near surface.

\subsection{Wavefield coherence}
Strong lower crustal heterogeneity with minimal mantle heterogeneity is not
adequate to match the general behavior of the observations.

\subsection{Discussion}
A stratified laminate appears to show transverse isotropy for low frequencies.
The quasi-laminar structure, the upper part of the mantle lithosphere,
will show transverse isotropy.

\subsection{Conclusions}
The presence os fine-scale heterogeneity in the crust and mantle makes
a major contribution to the nature of the coda of both P and S phases.

The change of heterogeneity style will have an effect on effective anisotropy
and may help contribute to the presence of
an apparent mid-lithosphere discontinuity.

% vim:sw=2:wrap
