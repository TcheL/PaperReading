%! TeX root = ../*.tex
\renewcommand{\pmk}{Belouchrani\_1998\_IEEEtsp\_Blind source separation}
\renewcommand{\prf}{Others/\pmk.pdf}
\renewcommand{\pti}{Blind source separation based on time-frequency
  signal representations}
\renewcommand{\pay}{Adel Belouchrani and Moeness G. Amin, 1998}
\renewcommand{\pjo}{IEEE Transactions on Signal Processing}
\renewcommand{\pda}{2020/9/13 Sun.}

\section{\pinfo}
\subsection{Introduction}
Blind source separation consists of recovering a set of singals
of which only instantaneous linear mixtures are observed.

\subsection{Spatial time-frequency distributions}
\myidxox{Other}{Method}{Time-frequency distributions (TFD)}
The discrete-time form of spatial time-frequency distributions (TFD)
for signal $ x(t) $, is given by
\[ D_{xx}(t, f) = \sum_{l = - \infty}^{\infty} \sum_{m = - \infty}^{\infty}
  \phi(m, l) x(t + m + l) \times x^{\ast}(t + m - l) e^{ - j 4 \pi f l} \]
where $t$ and $f$: the time and frequency index.
The kernel $\phi(m, l)$ characterizes the distribution and is a function
of both the time and lag variables.

The cross-TFD of two signals $ x_1(t) $ and $ x_2(t) $ is:
\[ D_{x_1 x_2}(t, f) = \sum_{l = - \infty}^{\infty} \sum_{m = - \infty}^{\infty}
  \phi(m, l) x_1(t + m + l) \times x_2^{\ast}(t + m - l) e^{ - j 4 \pi f l} \]

% vim:sw=2:wrap
