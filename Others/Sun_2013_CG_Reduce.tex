\renewcommand{\pmk}{Sun\_2013\_CG\_Reduce storage in RTM}
\renewcommand{\prf}{Others/\pmk.pdf}
\renewcommand{\pti}{Two effective approaches to reduce data storage
  in reverse time migartion}
\renewcommand{\pay}{Weijia Sun, Li-Yun Fu}
\renewcommand{\pjo}{Computers \& Geosciences}
\renewcommand{\pda}{2019/4/28 Sun.}

\section{\pinfo}
\subsection{Introduction}
\begin{enumerate}[\hspace{10mm}*]
  \item Reverse time migration: proposed by Baysal \etal, 1983; Whitmore, 1983.
  \item RTM for: VTI, Alkhalifah, 1998;
    TTI, Alkhalifah, 2000 \& Zhang and Zhang, 2009.
  \item RTM for elastic media: Yan and Sava, 2008; Yan and Xie, 2012.
  \item RTM for viscoelastic media: Deng and McMechan, 2008.
  \item \sline
  \item Optimal checkpointing for RTM: Symes, 2007.
  \item Checkpointing strategy for full waveform inversion:
    Andreson \etal, 2012.
  \item Pseudo-random boundary method: Clapp, 2009;
    Fletcher and Robertsson, 2011.
  \item \sline
  \item Reduce storage: in RTM, Symes, 2007; in FWI, Imbert \etal, 2011.
  \item Numerical algorithm:
    the finite-difference method, Etgen, 1986 \& Virieux, 1986;
    the pseudospectral method, Fornberg, 1987;
    the convolutional differentiator method, Zhou \etal, 1993;
    the Chebyshev expansion method, Pestana and Stoffa, 2010.
  \item Temporal interpolation for seismic records:
    Larner \etal, 1981; G\"ul\"unay, 2003.
  \item The sinc interpolation: Claerbout, 1985.
  \item Compression techniques in computer sciences: Salomon and Motta, 2010.
  \item The wavelet compression method in geophysics:
    Fomel and Liu, 2010; Wang \etal, 2010.
  \item \mynem{The DEFLATE algorithm}: Feldspar, 1997 (combine
    static Huffman coding: Huffman, 1952; with
    the LZ77 algorithm: Ziv and Lempel, 1977).
  \item The open-source package about the DEFLATE algorithm in C/C++:
    Gailly and Adler, 1995 (please click \href{http://www.zlib.net}{here}
    to download the package).
  \item \sline
  \item Data compression on GPUs: Nikitin \etal, 2011.
\end{enumerate}

The time step required by a numerical algorithm is generally much smaller
than the Nyquist time step.

\subsection{Reverse time migration}
The cross-correlation imaging condition for prestack depth migration:
\[ I(\mbf x) = \int_0^T S(\mbf x, t) R(\mbf x, t) dt \]
where $S(\mbf x, t)$: the source wavefields;
$R(\mbf x, t)$: the receiver wavefields.

\subsubsection{The Nyquist approach}
The Nyquist time step:
\[ \Delta t_{Nyq} = \frac{1}{2 (f_{max} - f_{min})} \]
where $f_{max}$ and $f_{min}$: the highest and lowest frequency
of seismic records.

The Nyquist approach \myidx{Other}{Method}{Nyquist approach}
stores the source wavefields and cross-correlates the source wavefields
with the receiver wavefields at every Nyquist time step,
and applies an anti-alias temporal interpolation
before extrapolating the receiver wavefields.

\paragraph{The sinc interpolation}
The sinc function:
\[ sinc(t) = \left\{
  \begin{array}{cc}
    \frac{sin(\pi f_N t)}{\pi f_N t} & t \neq 0 \\
    1                                & t = 0 \\
  \end{array}
\right. \]
where $f_N$: the Nyquist frequency, not smaller than $f_{max}$.
The reconstructed signal $f(t)$:
\[ f(t) = \sum_{n = 0}^N sinc(t - n\Delta t) u(n\Delta t) \]
where $N$: the sampling number.

\subsubsection{The compression approach}
Basic ways of data compression (Salomon and Motta, 2010):
the statistical method, the dictionary method, and the wavelet method.

\subsection{Conclusions}
Temporal interpolation of receiver wavefields extrapolation
should be performed to avoid high-frequency aliasing.

% vim:sw=2:wrap
