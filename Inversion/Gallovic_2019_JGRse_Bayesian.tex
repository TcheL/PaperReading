%! TeX root = ../*.tex
\renewcommand{\pmk}{Gallovic\_2019\_JGRse\_Bayesian dynamic inversion}
\renewcommand{\prf}{Inversion/\pmk.pdf}
\renewcommand{\pti}{Bayesian dynamic finite-fault inversion:
  1. method and synthetic test}
\renewcommand{\pay}{F. Gallovi\v{c}, \'L. Valentov\'a, J.-P. Ampuero,
  and A.-A. Gabriel, 2019}
\renewcommand{\pjo}{JGR: Solid Earth}
\renewcommand{\pda}{2019/11/20 Wed.}

\section{\pinfo}
\subsection{Introduction}
\begin{enumerate}[\hspace{10mm}*]
  \item Review of Monte Carlo methods in geophysical inversion:
    Sambridge and Mosegaard, 2002.
  \item \mynnem{Fd3d\_xy}
    \myidxox{Other}{Software}{Fd3d\_xy},
    dynamic rupture simulator based on a fourth-order staggered-grid
    velocity-stress method: Madariaga \etal, 1998 (please click
    \href{http://www.geologie.ens.fr/madariag/Programs/programs.html}{here}
    to download the code).
  \item \mynnem{WaveQLab3D}
    \myidxox{Other}{Software}{WaveQLab3D}:
    Duru and Dunham, 2016.
  \item The parallel tempering: Falcioni and Deem, 1999; Sambridge \etal, 2013.
  \item \sline
  \item Community benchmarks to assess the performance of earthquake-source
    inversioin methods: Mai \etal, 2016,
    \href{http://equake-rc.info/SIV}{Mainpage}.
  \item \mynnem{LinSlipInv}
    \myidxox{Other}{Software}{LinSlipInv},
    linear kinematic inversion: Gallovi\v{c} \etal, 2015.
\end{enumerate}

\subsection{Method}
\subsubsection{Bayesian formulation}
Denoting the prior PDF $p(\mbf m)$ of model parameters $\mbf m$ and
the PDF of data $\mbf d$ given the model parameters as $p(\mbf d | \mbf m)$,
the posterior PDF $p(\mbf m | \mbf d)$,
which is the solution of the inverse problem, reads
\[ p(\mbf m | \mbf d) = \frac{p(\mbf m) p(\mbf d | \mbf m)}{p(\mbf d)} \]
where the Bayesian evidence $p(\mbf d)$ serves as a normalization constants.
$p(\mbf m)$ is specified by some data-independent
constraints on the model parameters.
$p(\mbf d | \mbf m)$, is assumed to be Gaussian:
\[ p(\mbf d | \mbf m) = c_1 e^{ - \frac{1}{2} \sum_{i = 1}^N
  \frac{|| \mbf s_i(\mbf m) - \mbf d_i ||^2}{\sigma_i^2} } \]
where $\mbf d_i$ and $\mbf s_i(\mbf m)$: the observed data and synthetics
at station $i$; N: the number of stations; $||\cdot||$: the L2 norm;
$\sigma_i$: the assumed standard deviations representing
the combined uncertainty of the modeling and data errors.

\subsubsection{The posterior PDF}
In the parallel tempering method,
the posterior PDF is modified by a parameter temperature $T$:
\[ p(\mbf m | \mbf d, T) \myno{ = p(\mbf m) p(\mbf d | \mbf m, T) }
  = c_1 p(\mbf m) e^{ - \frac{1}{2 T} \sum_{i = 1}^N
  \frac{|| \mbf s_i(\mbf m) - \mbf d_i ||^2}{\sigma_i^2} } \]
and a set of Markov chains with different temperatures
advance through the model space.

% vim:sw=2:wrap
